\documentclass{article}
\usepackage[utf8]{inputenc}

\title{Sprawozdanie z Konfiguracji Docker-Compose.yml Projektu}
\author{Mykulskyi Mykyta, Vladyslav Kutsyn}
\date{\today}

\begin{document}

\maketitle

\section{Wstęp}
W niniejszym sprawozdaniu przedstawiam konfigurację pliku docker-compose.yml dla projektu oraz instrukcję, jak uruchomić ten projekt przy użyciu Dockera. Celem projektu było stworzenie środowiska do pracy z PHP, Laravel, Nginx, MySQL i Node.js. W ramach sprawozdania omówię kroki, jakie podjąłem, aby osiągnąć ten cel.

\section{Instrukcja uruchomienia projektu}
\subsection{Kroki do uruchomienia projektu}
Aby uruchomić projekt, postępuj zgodnie z poniższymi krokami:
\begin{enumerate}
    \item Zainstaluj Docker na swoim systemie, jeśli jeszcze tego nie zrobiłeś.
    \item Skopiuj plik docker-compose.yml do katalogu projektu.
    \item Otwórz terminal i przejdź do katalogu, w którym znajduje się plik docker-compose.yml.
    \item Uruchom projekt przy użyciu komendy:
    \begin{verbatim}
    docker-compose up -d
    \end{verbatim}
    \item Po zakończeniu procesu uruchomienia, projekt będzie dostępny pod adresem http://localhost.
\end{enumerate}

\subsection{Dostęp do konsoli PHP w kontenerze}
Aby uzyskać dostęp do konsoli PHP wewnątrz kontenera, wykonaj następujące kroki:
\begin{enumerate}
    \item Otwórz terminal.
    \item Uruchom poniższą komendę:
    \begin{verbatim}
    docker exec -it laravel-php bash
    \end{verbatim}
    \item Teraz jesteś wewnątrz kontenera PHP i możesz korzystać z narzędzi, takich jak Composer.
\end{enumerate}

\section{Opis pliku docker-compose.yml}
Poniżej znajduje się opis pliku docker-compose.yml:

\begin{verbatim}
% Wstaw plik docker-compose.yml tutaj
\end{verbatim}

\section{Opis poszczególnych usług}
\subsection{PHP with Laravel (php)}
Ta usługa używa obrazu php:8.0-fpm i jest odpowiedzialna za serwer PHP w połączeniu z Laravel. Pliki projektu są montowane z lokalnego katalogu ./laravel-app, a kontener pracuje w katalogu /var/www/html.

\subsection{Nginx (nginx)}
Usługa Nginx korzysta z obrazu nginx:latest i obsługuje serwer WWW. Port 80 jest mapowany na port 80 w kontenerze, a pliki konfiguracyjne Nginx są montowane z ./nginx-config.

\subsection{MySQL (mysql)}
Usługa MySQL używa obrazu mysql:5.7 i jest używana jako baza danych projektu. Skonfigurowane są odpowiednie zmienne środowiskowe i mapowane porty.

\subsection{Node.js (node)}
Ta usługa wykorzystuje obraz node:14 i służy do uruchamiania procesów budowy front-endu. Pliki projektu są montowane do kontenera, a wykonywane jest polecenie "npm install."

\section{Ewentualne uwagi}
W trakcie konfiguracji projektu warto zwrócić uwagę na wybór konkretnych obrazów Docker oraz ich wersji. Upewnij się, że używasz stabilnych i bezpiecznych obrazów, które są zgodne z wymaganiami Twojego projektu.

\section{Podsumowanie}
Podczas pracy nad projektem udało mi się skonfigurować środowisko Docker-Compose, które pozwala na uruchomienie projektu z PHP, Laravel, Nginx, MySQL i Node.js. Problemy napotkane podczas konfiguracji zostały rozwiązane. Jednak zawsze istnieje pole do poprawy, na przykład, można by dodać automatyczne skrypty do inicjalizacji bazy danych lub bardziej rozbudowane narzędzia do monitorowania kontenerów.

\end{document}
